\documentclass[11pt,a4paper]{ivoa}
\input tthdefs

\lstset{flexiblecolumns=true,numberstyle=\small,showstringspaces=False,
  identifierstyle=\texttt,defaultdialect=[latex]tex,language=tex}

\title{Operational Identification of Software Components in the Virtual
Observatory}

% see ivoatexDoc for what group names to use here
\ivoagroup{Ops}

\author[https://wiki.ivoa.net/twiki/bin/view/IVOA/WebHome?topic=MarkusDemleitner]{Markus
Demleitner}
\author[https://wiki.ivoa.net/twiki/bin/view/IVOA/MarkusDemleitner?topic=MarkTaylor]{Mark
Taylor}

\editor{Markus Demleitner}

% \previousversion[????URL????]{????Concise Document Label????}
\previousversion{This is the first public release}
       

\begin{document}
\begin{abstract}
In the Virtual Observatory, software often talks to other software, and
while doing it, it is often beneficial to know something about the
counterpart, be it for statistics (e.g., when ignoring validators) or
for operations (e.g., when enabling workarounds, or when such
workarounds can be dropped because components requiring them are no
longer in use).  This note collects recommended practices on both the
client and the server side.
\end{abstract}

\section*{Acknowledgement}

The bulk of this material was taken from a page on the IVOA
wiki\footnote{\url{https://wiki.ivoa.net/twiki/bin/view/IVOA/UserAgentUsage}}
that resulted from a discussion at the 2018 College Park Interop (client
side) and a
talk\footnote{\url{https://wiki.ivoa.net/internal/IVOA/InterOpOct2019Ops/serversoftware.pdf}}
at the 2019 Groningen Interop (server side).


\section*{Conformance-related definitions}

The words ``MUST'', ``SHALL'', ``SHOULD'', ``MAY'', ``RECOMMENDED'', and
``OPTIONAL'' (in upper or lower case) used in this document are to be
interpreted as described in IETF standard RFC2119 \citep{std:RFC2119}.

The \emph{Virtual Observatory (VO)} is a
general term for a collection of federated resources that can be used
to conduct astronomical research, education, and outreach.
The \href{http://www.ivoa.net}{International
Virtual Observatory Alliance (IVOA)} is a global
collaboration of separately funded projects to develop standards and
infrastructure that enable VO applications.


\section{Introduction}

Very early on in the construction of client-server architectures it was
found that it is useful to have mechanisms to find which software runs
at the other side of a connection, rather typcially to aid in debugging.
In particular, HTTP, which is at the basis of many of the VO's protocols
\citep{std:HTTP}, specifies the request header \emph{User-Agent} that
identifies the client software (``or statistical purposes, the tracing
of protocol violations, and automated recognition of user agents for the
sake of tailoring responses to avoid particular user agent
limitations'').

These general rules were found insufficient for various purposes in the
Virtual Observatory.  For instance, use case~\ref{uc:stats} was brought
up at the College Park Interop in 2018.  Also, it was found that most
servers did not identify the relevant protocol implementations but
rather libraries the services were built on or even reverse proxies they
ran behind -- information that is usually not useful when diagnosing
operational problems in the VO, which mostly originate in the higher
layers of request handling.

This note gives recommendations on how to make the header fields more
useful for VO operations.  While strictly only applicable to HTTP-based
protocols, we believe similar practices should be forseen where, as
typical in VOEvent transports, communication mechanisms are not built on
top of HTTP.


\subsection{Use Cases}

\subsubsection{Recognising Maintenance Queries}
\label{uc:stats}

In the VO, several clients connect services for operational purposes,
for instance in order to perform service validation or monitoring. If
service providers are gathering statistics on service usage, they may
wish to distinguish these different classes of request.

\subsubsection{Dropping Workarounds}

Client occasinally work around bugs in server software; these
workarounds over time are a maintenance liability, and hence it is
advantageous to drop them when they are no longer needed.  To find this
out, a simple way to enumerate server software running on the VO.

Conversely, a server may contain workarounds for client bugs.  Again,
being able to find out whether code implementing these can be safely
dropped without adverse consequences on clients actually in use has a
clear benefit.

\subsubsection{Client Snooping}

Servers may offer experimental or advanced features to clients it knows
(though such use should in general be frowned upon, as it violates the
spirit of interoperability).  Similarly, clients might under certain
circumstances want to enable or disable certain behaviour when realising
they are communicating with a known server component.

\subsubsection{Notifications}

As part of a responsible disclosure of a software weakness (or simply a
request for a software update), server developers might want to contact
deployers of vulnerable or otherwise broken software.

\subsubsection{Security and Privacy Considerations}

Several guidelines to IT security dissuade from giving details on the
software that drives a certain site in order to not give attackers
information that might be useful in an attack.

Following the practices proposed here will, indeed, weaken the
``security by obscurity'' put forward in these treatments.  On the other
hand, in the VO, where software providers rather typically are members
of the community, given the Registry, these software providers can contact
operators of vulnerable services given sufficiently precise software
identification even before a vulnerability is disclosed.

In that sense the server-side security implications of this proposal go
both ways.

The identification of the client has fewer security implications, as it
seems unlikely that services can target clients by software version.

Software identification does play a role in user privacy; user agents
are regularly employed in user tracking on the WWW.  While, presumably,
the generally non-profit operators in the VO will not use such data to
significantly violate their users' privacy, client authors may want to
give users the possibility to somewhat reduce the information content of
the headers proposed here.

On the other hand, the mechanisms proposed here are most relevant for
automated clients for which there usually are no privacy concerns.


\section{Client Identification}

\subsection{User-Agent Header Standard Usage}

The HTTP User-Agent header may be used by clients to identify their
nature or origin. The definiition and usage of this header is described
in RFC2616 \citep{std:HTTP} section 14.43, with additional text on
syntax in sections 3.8 and 2.2. The basic rule is that the content of
this field should consist of a sequence of tokens, where each token is
either a product name (with an optional version indicator), or a
free-text comment enclosed in parentheses. Formally (BNF from RFC2616):

\begin{lstlisting}
User-Agent        = "User-Agent" ":" 1*( product | comment )
product           = token ["/" product-version]
product-version   = token
comment           = "(" *( ctext | quoted-pair | comment ) ")"
ctext             = <any TEXT excluding "(" and ")">
\end{lstlisting}

Additional rules/conventions are that more-significant tokens should
appear earlier in the sequence, and that the content should be ``short
and to the point''.

\subsection{User-Agent Field IVOA Recommendations}

The Operations IG endorses and encourages use of these rules concerning
the User-Agent header, and adds a further convention, which does not
conflict with the above rules: clients whose primary purpose relates to
operations within the VO should indicate that purpose by including a
comment token of the form 
$$\hbox{\verb|(IVOA-<op-purpose> <optional-extra-text>)|.}$$

Suggested op-purpose values are currently:

\begin{description}
\item[validate]
client is performing service validation, e.g., to assess service
conformance to standards.
\item[monitor] client is performing service monitoring, e.g.,
to assess service availability or performance.
\item[harvest] client is harvesting records e.g., to populate 
a registry or registry-like database.
\end{description}

This list may evolve in the future (suggestions should be discussed on
the ops@ivoa.net mailing list). Custom op-purpose values are permitted.
Case is significant in op-purpose values and its "IVOA-" prefix.

The \verb|<optional-extra-text>| may  be used to indicate a URL at which
more information about the client, or perhaps about the results it is
gathering from the current request, can be found.

Formally:

\begin{lstlisting}
ivoa-comment  = "(IVOA-" op-purpose *( 
    ctext | quoted-pair | comment ) ")"
op-purpose   = "validate" | "monitor" | "harvest" | token
\end{lstlisting}

Tokens of the form \verb|ivoa-comment| should not appear in the
user-agent field if the request is a ``normal'' user science query. There
are obviously grey areas between operational and science requests; this
convention does not attempt to provide a rigid definition of these
categories.

This arrangement allows service operators to test in their logs for
User-agent values whose content matches the sequence \verb|"(IVOA-"|, or
perhaps \verb|"(IVOA-validate"|, and adjust their usage statistics
appropriately. Note, however, that it is not feasible to force operational
clients to follow this convention, so service operators will still need
to be careful in analysing their usage statistics.

\subsection{Examples}

A science query from the STILTS tapquery TAP client might contain the
HTTP header
\begin{lstlisting}
User-Agent: STILTS/3.1-4 Java/1.8.0_181
\end{lstlisting}
while a query from the STILTS taplint TAP service validator might
contain the header
\begin{lstlisting}
User-Agent: STILTS/3.1-4 (IVOA-validate) Java/1.8.0_181
\end{lstlisting}
or maybe (line break for typographic reasons)
\begin{lstlisting}
User-Agent: STILTS/3.1-4 (IVOA-validate
            http://validators.org/results) Java/1.8.0_181
\end{lstlisting}

\appendix
\section{Changes from Previous Versions}

No previous versions yet.  
% these would be subsections "Changes from v. WD-..."
% Use itemize environments.


\bibliography{ivoatex/ivoabib,ivoatex/docrepo}


\end{document}
